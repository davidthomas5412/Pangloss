% ****** Start of file apssamp.tex ******
%
%   This file is part of the APS files in the REVTeX 4.1 distribution.
%   Version 4.1r of REVTeX, August 2010
%
%   Copyright (c) 2009, 2010 The American Physical Society.
%
%   See the REVTeX 4 README file for restrictions and more information.
%
% TeX'ing this file requires that you have AMS-LaTeX 2.0 installed
% as well as the rest of the prerequisites for REVTeX 4.1
%
% See the REVTeX 4 README file
% It also requires running BibTeX. The commands are as follows:
%
%  1)  latex apssamp.tex
%  2)  bibtex apssamp
%  3)  latex apssamp.tex
%  4)  latex apssamp.tex
%
\documentclass[%
 reprint,
%superscriptaddress,
%groupedaddress,
%unsortedaddress,
%runinaddress,
%frontmatterverbose, 
%preprint,
%showpacs,preprintnumbers,
%nofootinbib,
%nobibnotes,
%bibnotes,
 amsmath,amssymb,
 aps,
%pra,
%prb,
%rmp,
%prstab,
%prstper,
%floatfix,
]{revtex4-1}

\usepackage{graphicx}% Include figure files
\usepackage{dcolumn}% Align table columns on decimal point
\usepackage{bm}% bold math
\usepackage{mathtools}
%\usepackage{multicol}
%\usepackage{hyperref}% add hypertext capabilities
%\usepackage[mathlines]{lineno}% Enable numbering of text and display math
%\linenumbers\relax % Commence numbering lines

%\usepackage[showframe,%Uncomment any one of the following lines to test 
%%scale=0.7, marginratio={1:1, 2:3}, ignoreall,% default settings
%%text={7in,10in},centering,
%%margin=1.5in,
%%total={6.5in,8.75in}, top=1.2in, left=0.9in, includefoot,
%%height=10in,a5paper,hmargin={3cm,0.8in},
%]{geometry}

\begin{document}

\preprint{APS/123-QED}

\title{SULI Report!}
%\thanks{A footnote to the article title}%

\author{Spencer Everett}
 \affiliation{Department of Physics, DePaul University.}%Lines break automatically or can be forced with \\
 \email{spencerweverett@gmail.com}

%\collaboration{MUSO Collaboration}%\noaffiliation

\date{\today}

\begin{abstract}
As dark matter does not absorb or emit light, its distribution in the universe must be inferred through indirect effects such as the gravitational lensing of distant galaxies. While most sources are only weakly lensed, the systematic alignment of background galaxies around a foreground lens can constrain the mass of the lens which is largely in the form of dark matter. In this paper, a simple halo model is used to approximate the dark matter mass distribution of the Millennium Simulation on the galaxy and cluster scales. Background sources are generated at a redshift of ${z = 1.3857}$ and then weakly lensed by the foreground halos contained in a light cone centered along each line of sight, which is carried out by the \texttt{Pangloss} framework\footnote{https://github.com/drphilmarshall/Pangloss}. The model-predicted ellipticities of the background galaxies after lensing are compared to the ``true'' weakly-lensed ellipticities determined by the ray-traced convergence and shear calculated by Hilbert et al. (citation). Using the shear-shear and galaxy-galaxy correlation functions, I find that the \texttt{Pangloss} framework systematically under-predicts the shear power in both statistics and does not accurately capture the effect of dark matter structure at scales larger than 1 arcminute. Physical and computational shortcomings of the \texttt{Pangloss} framework are discussed, as well as potential improvements for upcoming work.
\end{abstract}

%\pacs{Valid PACS appear here}% PACS, the Physics and Astronomy
                             % Classification Scheme.
%\keywords{Suggested keywords}%Use showkeys class option if keyword
                              %display desired
\maketitle

%\tableofcontents

\section{Introduction}

In a universe teeming with galaxies and light, it came as a shock when 20\textsuperscript{th} century astronomers unexpectedly discovered that most of the mass in the universe is in fact dark; the `normal' matter made of atoms that we interact with in everyday life, called baryonic matter, only accounts for (20\%?) of the mass in the observable universe. The remaining mass takes the form of an exotic dark matter that does not absorb or emit light, rendering it invisible to our telescopes. While this claim sounds bizarre, there has been an abundance of indirect evidence in recent decades for the existence of dark matter including the flattening of galaxy rotation curves (cite), the existence of baryonic acoustic oscillations (cite), and anisotropies in the cosmic microwave background (cite).

One of the most successful probes of dark matter has been gravitational lensing. The path of light from distant `background' galaxies is bent when traveling through regions of space containing large amounts of `foreground' mass. The process is described by Einstein's general theory of relativity, but simply put -- the more mass, the more light bends. Light from different origins in a background galaxy are subject to different bending which results in a distortion of the galaxy image. As the foreground mass is known to be largely dark matter, gravitational lensing supplies a direct constraint on the mass of dark matter in that region.

While the effects of gravitational lensing can be dramatic, the shape of most galaxies is only distorted by a few percent (cite) and can only be detected statistically as the intrinsic shape is not known. If a model for the distribution of dark matter in a region of foreground mass can accurately predict the statistical signal of this `weak' lensing of background galaxies, then the model can be used on galaxies in existing sky survey data to extrapolate the amount of dark matter in the region and construct large-scale maps of the dark matter in the universe.

In this paper, I aim to do the former by applying a simple dark matter halo model to the Millennium Simulation and predicting the weakly lensed ellipticities of generated background sources. A brief introduction into the theory of galaxy ellipticities, the effects of strong and weak gravitational lensing, and dark matter halos is discussed in Section II. The implementation of a halo model on the Millennium Simulation is described in Section III, and the results of the model on (x \# of galaxies on a field size of x \# deg$^2$) as well as a comparison of the predicted lensed ellipticities to the true ellipticities is given in Section IV.  Section V discusses limitations of the used Pangloss framework as well as potential physical and computational improvements that can be made for upcoming work before concluding remarks in Section VI.

\section{Theoretical Background}

\subsection*{Galaxies as Ellipses}
Consider a galaxy image that can be well approximated as an ellipse at an angle $\phi$ above an arbitrary reference line. The galaxy's complex ellipticity is defined to be 
\begin{equation}
\varepsilon=\varepsilon_1+i\varepsilon_2=|\varepsilon|\,e^{2i\phi}
\end{equation}

\noindent where the magnitude of the galaxy's ellipticity $|\varepsilon|$ is defined to be
\begin{equation}
|\varepsilon|=\frac{1-r}{1+r}
\end{equation}

\noindent and $r\leq1$ is the ratio of the semi-minor and semi-major axis of the ellipse. This compact notation combines the eccentricity and orientation of the ellipse into a single object. A plot from (Schneider et al.) showing the shape of elliptical galaxies for various values of $\varepsilon_1$ and $\varepsilon_2$ is shown in Figure (??).

%************************
(Galaxy Ellipticity Figure)
%************************

(Mention shape nosie and PSF here?)

\subsection*{Gravitational Lensing}

A full mathematical treatment of the gravitational lensing of galaxies due to the gravitational fields of massive objects requires general relativity (see (Beck cite) for details). However, the important results can be summarized as follows. Foreground mass distorts the image of a background galaxy in two distinct ways: The image is magnified and sheared tangentially about the foreground mass, making it more elliptical. The magnification of the image is determined by the convergence $\kappa$, a scalar which measures the projected mass density along each line of sight. The shearing of the source is most often described by the complex shear $\gamma$ defined to be
\begin{equation}
\gamma=\gamma_1+i\gamma_2=|\gamma|\,e^{2i\varphi}
\end{equation}

\noindent where $|\gamma|$ is the magnitude of the shear and $\varphi$ is the orientation of the shear. While the intrinsic ellipticity of the source may be in any orientation near the foreground mass, it will be more aligned with the shear field after lensing. Figure (??) demonstrates this process visually.

However, usually the quantity of interest in lensing calculations is the \textit{reduced} shear, defined by
\begin{equation}
g=\frac{\gamma}{1-\kappa}.
\end{equation}

\noindent Then using the thin lens approximation for the lensing of a background source of intrinsic ellipticity $\varepsilon_i$ around a point foreground mass with reduced shear $g$, the lensed ellipticity $\varepsilon$ is given by
\begin{equation}
 \varepsilon = \begin{dcases} 
      \frac{\varepsilon_i+g}{1+g^*\varepsilon_i} & : |g|\leq1; \\[0.75em]
       \frac{1+g\varepsilon_i^*}{\varepsilon^*+g^*} & : |g|>1.
   \end{dcases}
\end{equation}

\noindent where an asterisk denotes the complex conjugate (cite Schneider). The behavior of the distortion relies strongly on the magnitude of $g$; the
effect is called \textit{strong} lensing if $|g|>1$ and \textit{weak} lensing if $|g|<1$. The effects of strong lensing can be quite dramatic, distorting sources into large arcs, multiple images, or even a complete Einstein ring shown in Figure (??). While strong lenses are rare as the alignment of the source and foreground mass must be nearly perfect, \textit{all} sources are weakly lensed. The effect is small, usually an ellipticity distortion of only a few percent (cite?), but can be detected locally by averaging the ellipticities of all sources in a small region. As the orientations of the sources should be random, it would be expected that
$$\left<\varepsilon\right>=0.$$

\noindent However as sources in the same small region are sheared in (approximately) the same way, this implies that
\begin{equation}
\left<\varepsilon\right>=g.
\end{equation}

 \noindent Finally as in the weak lensing regime ${\kappa\ll1}$ and ${|\gamma|\ll1}$, it follows that
 \begin{equation}
 \gamma\approx g=\left<\varepsilon\right>
 \end{equation}

 \noindent which provides a method of detecting the shear observationally.

\subsection*{Dark Matter Halos}

While the exact relation between the distribution of galaxies and dark matter is not known, simulations have shown that galaxies tend to form in overdense regions of dark matter. This means that galaxies \textit{should} trace out the larger underlying dark matter structures. The simplest way to model this relationship is by enveloping each galaxy in a spherically symmetric dark matter `halo' of mass $M_h$ sampled from the stellar-to-halo mass relation (cite SMHR). These halos extend far beyond the edge of the visible galaxy that they enclose. While the density profile of the halos may be complex, numerous simulations have shown that it can be well approximated by the Navarro-Frenk-White (NFW) profile of the form
\begin{equation}
\rho_{NFW}(r)=\frac{\rho_0}{\frac{r}{R_s}\left(1+\frac{r}{R_s}\right)^2}
\end{equation}

\noindent where the constant $\rho_0$ and the scale radius $R_s$ are parameters that vary from halo to halo (cite NFW). This work uses a truncated NFW profile called the Baltz-Marshall-Oguri (BMO) profile given by
\begin{equation}
\rho_{BMO}(r)=\rho_{NFW}\cdot\left(\frac{r_t^2}{r^2+r_t^2}\right)^2
\end{equation}

\noindent where $r_t$ is a free parameter, as it has been shown to be a better fit to simulated data (cite BMO).

\section{The Pangloss Framework}

To constrain the mass of foreground dark matter using weak gravitational lensing, first a model of the relationship between foreground galaxies and the foreground dark matter must be established and robustly tested to see if, statistically, it makes the same lensing predictions of background sources as the true undering dark matter structure. To do this, I built upon the publically available \texttt{Pangloss} framework used in Collett et al. (cite) to reconstruct all the mass along the line of sight of each background galaxy using dark matter halos. The lensing contribution of each halo is calculated, and the total lensing of the background galaxy is the sum of each halo contribution. This process is detailed in the following sections.

\subsection*{Assumptions}
While \texttt{Pangloss} may be used more generally, the present analysis makes some additional strong assumptions to simplify the problem for a first attempt at making weak lensing predictions.

\begin{enumerate}
\item The dark matter mass distribution can be approximated by spherically symmetric BMO halo profiles attached to each galaxy.
\item The stellar mass of the foreground galaxy is negligible for lensing calculations.
\item The mass of the dark matter halo of each foreground galaxy is known.
\item A spectroscopic redshift of each foreground galaxy is known.
\end{enumerate}

Testing the first assumption is the main goal of this paper. The second assumption is reasonable as it is estimated that dark matter halos are 1-2 orders of magnitude more massive than the host galaxy (cite). Clearly the third assumption will never be true for any observational data. However, this allows for a best-case scenario estimate of how well the \texttt{Pangloss} framework \textit{could} predict weak lensing effects given all possible information. This assumption can be relaxed by sampling a dark matter halo mass from an assumed stellar mass - halo mass relation (cite). The fourth assumpition is also unrealistic as most galaxies in sky surveys only have a less-reliable photometric redshift due to time constraints, but this again allows for a best-case estimate. This assumption could be relaxed by adding random photometric redshift noise and repeating the upcoming analysis on many realizations of the photometric redshifts.

\subsection*{The Millennium Simulation}
\texttt{Pangloss} cannot be used to make dark matter mass maps using existing galaxy catalogs until it is tested on a simulated universe with known dark matter structure to determine how accurately and precisely it predicts the lensing of background sources. For this purpose, \texttt{Pangloss} was tested on galaxy catalogs from the Millennium Simulation, an N-body simulation consisting of over 10 billion dark matter `particles' each representing a billion solar masses and populated with about 20 million galaxies (cite 2005). The simulation uses cosmological parameters from WMAP 1\textsuperscript{st}-year data analysis and contains baryonic and dark matter structure believed to be consistent with our own universe. From the work of Hilbert et al. (cite), high resolution maps of the shear and convergence from the perspective of a single reference point calculated by ray-tracing are publically available. From these maps, the actual lensing of background galaxies when traveling through the foreground mass of the Millennium Simulation towards the reference point can be calculated using Equation (??).

\subsection*{Generating Background Galaxies}
With a catalog of foreground galaxies chosen, a set of background galaxies with density 10 per square arcminute was generated. The intrinsic orientation of each galaxy was sampled from a uniform distribution as, without lensing, there should be no preferred orientation due to the assumption of an isotropic universe. The magnitude of the galaxy ellipticities was sampled from a normal distribution with a standard deviation of 0.2, but any ellipticities with magnitude greater than one were re-sampled. Random ellipticity noise was sampled from a normal distribution with a standard deviation of 0.1 and added to the intrinsic ellipticities. Finally, each ellipticity was multiplied by 0.9 to account for systematic shape noise of 10\%.

\subsection*{Creating Lightcones}
While ideally all foreground mass in a field would be considered when predicting the weak lensing of a background galaxy, it is computationally prohibitive to do so. Instead, all foreground halos contained within a `lightcone' centered along the line of sight to the source and extending out to a chosen lightcone radius $R$ were considered when calculating lensing contributions for the background source. A cartoon model of this process is shown in Figure (???). Unless otherwise specified, experiments in this paper used a radius of 2 arcminutes.

%************************
(Pangloss Cartoon Figure)
%************************

To calculate the convergence and shear contribution of each halo, the physical distance from the halo to the line of sight was needed. To increase the speed of distance calculations, the foreground halo redshifts were first snapped to a grid of 100 equally-sized redshift bins and then converted to physical distance using (still trying to decipher the grid.py/distance.py code. Ask Phil!).

%************************
(Redshift-to-distance relation equation?)
%************************

Using the physical separation distance and halo mass, the shear and convergence of a single foreground halo is calculated using methods described in Wright and Brainerd (cite). The total convergence and shear at the center of the lightcone is simply the sum of the convergence and shear contributions of each halo contained in the lightcone. The \texttt{Pangloss}-predicted lensed background ellipticity is then calculated using Equation (??).

\subsection*{Checking the Halo Model}

Instead of comparing the ray-traced and \texttt{Pangloss}-predicted lensed ellipticities for individual galaxies, the lensing is characterized globally with correlation functions. The ellipticity-ellipticity correlation function measures how correlated the ellipticities of pairs of galaxies are as a function of separation distance, while the galaxy-galaxy correlation function measures the correlation of lensed ellipticities around foreground halos as a function of separation distance. For readers that are unfamiliar with correlation functions in the context of cosmology or want a visual aid, see Appendix A. Both correlation functions are used in this work to measure how well the \texttt{Pangloss} framework models weak lensing by dark matter structures using the publically available \texttt{TreeCorr} module written by Mike Jarvis\footnote{https://github.com/rmjarvis/TreeCorr}. Note that the correlation function definition used in \texttt{TreeCorr} is slightly different than that used in most of the literature; for a derivation of the connection between Jarvis's definition and the more common Schneider definition (cite), see Appendix B.

\section{Results}

\begin{itemize}
\item Maps??
\item Correlation Functions
    \begin{itemize}
    \item shear-shear
    \item galaxy-galaxy
    \item progression of each as $R$ increases??
    \item cf`s without shape noise and with shape noise
    \item cf`s with all galaxies vs with only `relevant' galaxies?
    \end{itemize}
\item The Need for Speed
    \begin{itemize}
    \item Table or graph of speedups and CPU time per lightcone progression (make sure to check \texttt{timeit} first!)
    \end{itemize}
\end{itemize}

\section{Discussion}

\begin{itemize}
\item Issues with the \texttt{Pangloss} framework
    \begin{itemize}
    \item Systematic shear power under-prediction in cf`s
    \item Missing large-scale dark matter structure
    \end{itemize}
    \item Possibly missing features in model
        \begin{itemize}
        \item voids
        \item filaments
        \item stellar mass
        \end{itemize}
    \item Speed
\item Moving Foward
    \begin{itemize}
    \item Scaling up
        \begin{itemize}
        \item CPU and RAM numbers for lightcone parallelization from discussion with Debbie Bard
        \end{itemize}
    \item infer hyper-parameters from SMHR, physical distance, etc
    \item other stuff - too tired right now
    \end{itemize}
\end{itemize}

\section{Conclusion}

pass

\section{Acknowledgements}

\begin{itemize}
\item DOE
\item SULI
\item Stanford University
\item KIPAC
\item Risa, Matt, Pat
\item Phil!!
\end{itemize}

\section{References}

\begin{itemize}
\item P.J. Marshall Thesis?
\item Collett et al.
\item S. Hilbert (ray-traced) http://arxiv.org/abs/0809.5035
\item Millennium Simulation
\item Schneider
\item Jarvis \texttt{TreeCorr}
\item lensing calculation http://arxiv.org/pdf/astro-ph/9908213v1.pdf
\item NFW profile
\item BMO truncation, BMO vs NFW http://arxiv.org/pdf/1301.1684v2.pdf
\item SMHR http://arxiv.org/pdf/1401.7329v1.pdf
\item Planck
\item galaxy curves
\item BAOs
\item LCDM
\item GR
\item wl distortion percent
\end{itemize}

\onecolumngrid
\section{Appendix}

\subsection{Visualizing Correlation Functions}

\begin{itemize}
\item Use as a brief introduction to the shear-shear and galaxy-galaxy correlation function as used in this work
    \begin{itemize}
    \item reference other (much more detailed) material for readers to find
    \end{itemize}
\item Include correlation value colormaps for $\xi_+$, $\xi_-$, and $\xi_\times$ along with discussion of correlation pattern
\item Include color-coded correlation visuals along with setup
    \begin{itemize}
    \item Galaxies around lens before lensing - correlation values scatter plot with no signal
    \item Galaxies around lens after lensing - correlation values scatter plot with signal
    \end{itemize}
\item B-modes and E-modes plot??
\end{itemize}

\subsection{Jarvis Proof}

\begin{itemize}
\item Introduction to difference between the common Schneider definition of lensing correlation functions and that used by Jarvis`s \texttt{TreeCorr} code
\item Here we derive the connection between the two
\item Definitions of each, along with references
\end{itemize}

From Jarvis \href{http://arxiv.org/pdf/astro-ph/0307393v2.pdf}{(Page 3)}:
\begin{align}
\xi_+&=\left<\gamma_i\gamma_j^*\right>=\texttt{xip}+i(\texttt{xip\_im})\label{s+}\\
\xi_-&=\left<\gamma_i\gamma_je^{-4i\alpha}\right>=\texttt{xim}+i(\texttt{xim\_im})\label{s-}
\end{align}

where $\alpha$ is the angle between the two objects $i,j$ and each $\gamma$ is given by $\gamma_n=|\gamma_n|e^{2i\theta_n}$ in polar form.\\

From Schneider \href{http://arxiv.org/pdf/astro-ph/0509252v1.pdf}{(Page 92)}:
\begin{align}
\xi_\pm&=\left<\gamma_{i_t}\gamma_{j_t}\right>\pm\left<\gamma_{i_\times}\gamma_{j_\times}\right>\label{s+-}\\
\xi_\times&=\left<\gamma_{i_t}\gamma_{j_\times}\right>\label{sx}
\end{align}

where $\gamma_{n_t}=-\operatorname{Re}\left(\gamma_n e^{-2i\alpha}\right)$ and $\gamma_{n_\times}=-\operatorname{Im}\left(\gamma_n e^{-2i\alpha}\right)$.\\

\section*{Schneider's $\xi_+$ to Jarvis's \texttt{xip}}

Starting with Schneider's definition, observe that
\begin{align*}
\xi_+&=\left<\gamma_{i_t}\gamma_{j_t}\right>+\left<\gamma_{i_\times}\gamma_{j_\times}\right>\\
&=\left<\operatorname{Re}\left(\gamma_ie^{-2i\alpha}\right)\cdot\operatorname{Re}\left(\gamma_je^{-2i\alpha}\right)\right>+\left<\operatorname{Im}\left(\gamma_ie^{-2i\alpha}\right)\cdot\operatorname{Im}\left(\gamma_je^{-2i\alpha}\right)\right>\\
&=\left<\operatorname{Re}\left(|\gamma_i|e^{2i(\theta_i-\alpha)}\right)\cdot\operatorname{Re}\left(|\gamma_j|e^{2i(\theta_j-\alpha)}\right)\right>+\left<\operatorname{Im}\left(|\gamma_i|e^{2i(\theta_i-\alpha)}\right)\cdot\operatorname{Im}\left(|\gamma_j|e^{2i(\theta_j-\alpha)}\right)\right>\\
&=\left<|\gamma_i|\cdot|\gamma_j|\cos\left(2(\theta_i-\alpha)\right)\cos\left(2(\theta_j-\alpha)\right)\right>+\left<|\gamma_i|\cdot|\gamma_j|\sin\left(2(\theta_i-\alpha)\right)\sin\left(2(\theta_j-\alpha)\right)\right>.
\end{align*}

Using the trig identities
\begin{align}
\cos(u)\cos(v)&=\frac{1}{2}\left[\cos(u-v)+\cos(u+v)\right],\\
\sin(u)\sin(v)&=\frac{1}{2}\left[\cos(u-v)-\cos(u+v)\right],
\end{align}

and the linearity of the expectation operator, it is straightforward to show that
%\begin{align*}
%\xi_+&=\left<|\gamma_i|\cdot|\gamma_j|\right>\cdot\left(\left<\frac{1}{2}\left[\cos\left(2(\theta_i-\theta_j)\right)+\cos\left(2(\theta_i+\theta_j-2\alpha)\right)\right]\right>+\left<\frac{1}{2}\left[\cos\left(2(\theta_i-\theta_j)\right)-\cos\left(2(\theta_i+\theta_j-2\alpha)\right)\right]\right>\right)\\
%&=\left<|\gamma_i|\cdot|\gamma_j|\cos\left(2(\theta_i-\theta_j)\right)\right>.
%\end{align*}
$$\xi_+=\left<|\gamma_i|\cdot|\gamma_j|\cos\left(2(\theta_i-\theta_j)\right)\right>.$$

Now using the identity
\begin{equation}\label{trig2}
\cos(u\pm v)=\cos\left(u\right)\cos\left(v\right)\mp\sin\left(u\right)\sin\left(v\right),
\end{equation}

the previous equation can be written as

\begin{align*}
\xi_+&=\left<|\gamma_i|\cdot|\gamma_j|\cdot\left[\cos\left(2\theta_i\right)\cos\left(2\theta_j\right)+\sin\left(2\theta_i\right)\sin\left(2\theta_j\right)\right]\right>\\
&=\left<\operatorname{Re}(\gamma_i)\cdot\operatorname{Re}(\gamma_j)+\operatorname{Im}(\gamma_i)\cdot\operatorname{Im}(\gamma_j)\right>.
\end{align*}

Using the complex number identities

\begin{align}
\operatorname{Re}(z)&=\frac{1}{2}\left[z+z^*\right],\label{complex1}\\
\operatorname{Im}(z)&=\frac{1}{2i}\left[z-z^*\right]\label{complex2},
\end{align}

it follows that

\begin{align*}
\xi_+&=\left<\frac{1}{4}\left[(\gamma_i+\gamma_i^*)(\gamma_j+\gamma_j^*)-(\gamma_i-\gamma_i^*)(\gamma_j-\gamma_j^*)\right]\right>\\
&=\left<\frac{1}{2}\left[\gamma_i\gamma_j^*+(\gamma_i\gamma_j^*)^*\right]\right>\\
&=\left<\operatorname{Re}(\gamma_i\gamma_j^*)\right>=\texttt{xip}.
\end{align*}

\section*{Schneider's $\xi_-$ to Jarvis's \texttt{xim}}

The first few steps are identical to the previous section besides the negative in the definition of $\xi_-$, giving
\begin{align*}
\xi_-&=\left<\gamma_{i_t}\gamma_{j_t}\right>-\left<\gamma_{i_\times}\gamma_{j_\times}\right>\\
&=\left<|\gamma_i|\cdot|\gamma_j|\cos\left(2(\theta_i-\alpha)\right)\cos\left(2(\theta_j-\alpha)\right)\right>-\left<|\gamma_i|\cdot|\gamma_j|\sin\left(2(\theta_i-\alpha)\right)\sin\left(2(\theta_j-\alpha)\right)\right>\\
&=\left<|\gamma_i|\cdot|\gamma_j|\cos\left(2(\theta_i+\theta_j-2\alpha)\right)\right>.
\end{align*}

Now using Equation \eqref{trig2} twice, first letting $u=2(\theta_i+\theta_j)$ and $v=-4\alpha$, and then letting $u=2\theta_i$ and $v=2\theta_j$, this equation becomes

\begin{align*}
\xi_-&=\left<|\gamma_i|\cdot|\gamma_j|\Big(\left[\cos(2\theta_i)\cos(2\theta_j)-\sin(2\theta_i)\sin(2\theta_j)\right]\cos(4\alpha)+\left[\sin(2\theta_i)\cos(2\theta_j)+\cos(2\theta_i)\sin(2\theta_j)\right]\sin(4\alpha)\Big)\right>\\
&=\Big<\big[\operatorname{Re}(\gamma_i)\cdot\operatorname{Re}(\gamma_j)-\operatorname{Im}(\gamma_i)\cdot\operatorname{Im}(\gamma_j)\big]\cos(4\alpha)+\big[\operatorname{Im}(\gamma_i)\cdot\operatorname{Re}(\gamma_j)+\operatorname{Re}(\gamma_i)\cdot\operatorname{Im}(\gamma_j)\big]\sin(4\alpha)\Big>.
\end{align*}

Using the identities \eqref{complex1} and \eqref{complex2}, and simplifying the leftover terms, this equation can be shown to equal
\begin{align*}
\xi_-=\big<\operatorname{Re}(\gamma_i\gamma_j)\cos(4\alpha)+\operatorname{Im}(\gamma_i\gamma_j)\sin(4\alpha)\big>.
\end{align*}

Now observe that for two complex numbers $a$ and $b$, it is true that

$$\operatorname{Re}(a\cdot b^*)=\operatorname{Re}(a)\cdot\operatorname{Re}(b)+\operatorname{Im}(a)\cdot\operatorname{Im}(b).$$

Then setting $a=\gamma_i\gamma_j$ and $b=e^{4i\alpha}$, it must be true that

$$\operatorname{Re}\left(\gamma_i\gamma_je^{-4i\alpha}\right)=\operatorname{Re}(\gamma_i\gamma_j)\cos(4\alpha)+\operatorname{Im}(\gamma_i\gamma_j)\sin(4\alpha).$$

Combining this with our previous result, this means that

$$\xi_-=\big<\operatorname{Re}(\gamma_i\gamma_j)\cos(4\alpha)+\operatorname{Im}(\gamma_i\gamma_j)\sin(4\alpha)\big>=\left<\operatorname{Re}\left(\gamma_i\gamma_je^{-4i\alpha}\right)\right>=\texttt{xim}.$$

\section*{Schneider's $\xi_\times$ to Jarvis's $\frac{1}{2}\left(\texttt{xim\_im}-\texttt{xip\_im}\right)$}

Starting from Schneider's definition of $\xi_\times$,

\begin{align*}
\xi_\times&=\left<\gamma_{i_t}\gamma_{j_\times}\right>\\
&=\left<\operatorname{Re}\left(|\gamma_i|e^{2i(\theta_i-\alpha)}\right)\cdot\operatorname{Im}\left(|\gamma_j|e^{2i(\theta_j-\alpha)}\right)\right>\\
&=\big<|\gamma_i|\cdot|\gamma_j|\cos\left(2(\theta_i-\alpha)\right)\sin\left(2(\theta_j-\alpha)\right)\big>.\\
\end{align*}

Using the trig identity

\begin{equation}\label{trig3}
\cos(u)\sin(v)=\frac{1}{2}\left[\sin(u+v)-\sin(u-v)\right],
\end{equation}

the previous equation becomes

\begin{align*}
\xi_\times&=\left<|\gamma_i|\cdot|\gamma_j|\cdot\frac{1}{2}\big[\sin\left(2(\theta_i+\theta_j-2\alpha)\right)-\sin\left(2(\theta_i-\theta_j)\right)\big]\right>.
\end{align*}

Next, applying the identities

\begin{equation}\label{trig4}
\sin(u\pm v)=\sin(u)\cos(v)\pm\cos(u)\sin(v)
\end{equation}

along with \eqref{trig2} iteratively until each trig function only has a single term, the equation becomes

\begin{align*}
\xi_\times&=\Big<|\gamma_i|\cdot|\gamma_j|\cdot\frac{1}{2}\Big(\left[\sin(2\theta_i)\cos(2\theta_j)+\cos(2\theta_i)\sin(2\theta_j)\right]\cos(4\alpha)\\
&\qquad\qquad\qquad\quad-\left[\cos(2\theta_i)\cos(2\theta_j)-\sin(2\theta_i)\sin(2\theta_j)\right]\sin(4\alpha)\\
&\qquad\qquad\qquad\quad-\sin(2\theta_i)\cos(2\theta_j)+\cos(2\theta_i)\sin(2\theta_j)\Big)\Big>\\
&=\Big<\frac{1}{2}\Big(\left[\operatorname{Im}(\gamma_i)\cdot\operatorname{Re}(\gamma_j)+\operatorname{Re}(\gamma_i)\cdot\operatorname{Im}(\gamma_j)\right]\cos(4\alpha)\\
&\qquad-\left[\operatorname{Re}(\gamma_i)\cdot\operatorname{Re}(\gamma_j)-\operatorname{Im}(\gamma_i)\cdot\operatorname{Im}(\gamma_j)\right]\sin(4\alpha)\\
&\qquad-\operatorname{Im}(\gamma_i)\operatorname{Re}(\gamma_j)+\operatorname{Re}(\gamma_i)\cdot\operatorname{Im}(\gamma_j)\Big)\Big>.\\
\end{align*}

Using the identities \eqref{complex1} and \eqref{complex2}, this can be simplified to

\begin{align*}
\xi_\times&=\left<\frac{1}{2}\big[\operatorname{Im}(\gamma_i\gamma_j)\cos(4\alpha)-\operatorname{Re}(\gamma_i\gamma_j)\sin(4\alpha)-\operatorname{Im}(\gamma_i\gamma_j^*)\big]\right>.
\end{align*}

Consider again two complex numbers $a$ and $b$. Observe that

$$\operatorname{Im}(a\cdot b^*)=\operatorname{Im}(a)\cdot\operatorname{Re}(b)-\operatorname{Re}(a)\cdot\operatorname{Im}(b).$$

Then setting $a=\gamma_i\gamma_j$ and $b=e^{4i\alpha}$, it must be true that

$$\operatorname{Im}\left(\gamma_i\gamma_je^{-4i\alpha}\right)=\operatorname{Im}(\gamma_i\gamma_j)\cos(4\alpha)-\operatorname{Re}(\gamma_i\gamma_j)\sin(4\alpha).$$

Combing these results give

\begin{align*}
\xi_\times&=\left<\frac{1}{2}\big[\operatorname{Im}(\gamma_i\gamma_j)\cos(4\alpha)-\operatorname{Re}(\gamma_i\gamma_j)\sin(4\alpha)-\operatorname{Im}(\gamma_i\gamma_j^*)\big]\right>\\
&=\left<\frac{1}{2}\big[\operatorname{Im}\left(\gamma_i\gamma_je^{-4i\alpha}\right)-\operatorname{Im}(\gamma_i\gamma_j^*)\big]\right>\\
&=\frac{1}{2}\big[\texttt{xim\_im}-\texttt{xip\_im}\big].
\end{align*}

as desired.

% Bibliography

\end{document}
%
% ****** End of file apssamp.tex ******

\end{document}

\begin{itemize}

\end{itemize}



% Outline:

%% Introduction:
\begin{itemize}
\item Broad intro into cosmology and dark matter, set problem in context
    \begin{itemize}
    \item (Very) short introduction to history of dark matter, Zwicky, rotation curves, etc.
    \end{itemize}
\begin{itemize}
\item Introduction to gravitational lensing
    \begin{itemize}
    \item Convergence and Shear / Strong and Weak lensing
        \begin{itemize}
        \item Dramatic effect of strong lensing, how it is used for constraining cos. parameters
        \item Only a few hundred known, not good for making mass maps
        \item All galaxies are weakly lensed, good for our purposes
        \end{itemize}
    \end{itemize}
\end{itemize}
\item What this work does and aims for
\begin{itemize}
    \item We aim to implement a simple dark matter halo model to predict weak-lensing effects by foreground mass - and to make the prediction `quickly' and/or with large parallelization
    \item Quick summary of what we do in paper
        \begin{itemize}
        \item Use halo model to make approximation of dark matter mass distribution in Millennium Simulation
        \item Use shear-shear and galaxy-galaxy correlation functions as summary statistics
        \item Compare statistics to ray-traced data
        \item Discuss physial and computational shortcomings with current methodology
        \end{itemize}
    \end{itemize}
\end{itemize}

%% Theoretical Background:
\begin{itemize}
\item Galaxy ellipticities
    \begin{itemize}
    \item (Most) galaxies are not circular - have intrinsic ellipticities
        \begin{itemize}
        \item If they were all circular, we could measure the change in shape exactly!
        \end{itemize}
    \item Galaxies are not always well approximated by ellipses (reference)
    \item Mathematical formalism
    \item Plot of galaxy orientation given e1 and e2 from P.J. Marshall thesis
    \end{itemize}
\item Shape noise and PSF issues in weak lensing (distorting already weak signal)
\item Gravitational lensing
    \begin{itemize}
    \item Short introduction to problem setup for simple case (point mass maybe?)
    \item Assumptions made in a typical solution (thin lens approximation?)
    \item Show solution from Schneider
    \item Shear and convergence 
    \item Measuring shear in weak lensing - a statistical feature
    \end{itemize}
\item Dark matter halos
    \begin{itemize}
    \item NFW profile and equation (reference)
    \item BMO truncation and equation (reference)
    \end{itemize}
\item SMHR? Or too distracting?
\end{itemize}

% Old theory:
%However, the important results for a point mass using the thin lens approximation can be summarized as follows: The distortion of a background galaxy image can be quantified by the linearized lens mapping Jacobi matrix $\mathcal{A}$ which can be written as

%$$\mathcal{A}=(1-\kappa)\,
%\begin{pmatrix}
%1-g_1 & -g2\\
%-g_2 & 1+g_1
%\end{pmatrix}$$

%\noindent where $\kappa$ and $g$ are the convergence and reduced shear respectively at the center of the background source. 
%\begin{displaymath}
%  \varepsilon = \left\{
%    \begin{array}{lr}
%      \frac{\varepsilon_i+g}{1+g^*\varepsilon_i} & : |g|\leq1; \\
%      \frac{1+g\varepsilon_i^*}{\varepsilon^*+g^*} & : |g|>1.
%    \end{array}
%   \right.
%\end{displaymath}


%% The Pangloss Framework:
\begin{itemize}
\item Assumptions
    \begin{itemize}
    \item A catalog with galaxy
    \item Currently use actual dark matter halo mass for an idea of best case scenario - can run same analysis with masses sampled from SMHR (reference)
    \item PSF already accounted for in catalog and error low enough for weak lensing analysis
    \item Use actual redshift for best case scenario (but can easily be replaced by photometric redshift sampling)
    \end{itemize}
\item Millennium Simulation
    \begin{itemize}
    \item Instead of blindly using \texttt{Pangloss} to make dark matter mass maps using sky survey data, we instead test how accurately and precisely we can predict weakly-lensed shear and convergence 
    \item Mention which M.S. we used
    \item S. Hilbert`s ray traced M.S. (reference)
        \begin{itemize}
        \item Can use the ray-traced convergence and shear to calculate the ``true'' lensed ellipticties and compare to \texttt{Pangloss}
        \end{itemize}
    \end{itemize}
\item Background galaxies
    \begin{itemize}
    \item random orientation
    \item ellipticity range (get numbers!)
    \item Accounting for shape noise
    \end{itemize}
\item Lightcones
    \begin{itemize}
    \item Instead of using all foreground mass in a field to calculate the lensing of a single background object, we create a lightcone centered at the source with a chosen radius $R$
    \item Drilling
        \begin{itemize}
        \item redshift grid to speed up calcuations
        \item Use the physical distance to line of sight in lensing calculations
            \begin{itemize}
            \item Redshift-to-distance relation (equation and reference)
            \end{itemize}
        \item Ranking galaxy `relevance'
            \begin{itemize}
            \item Want to only include the most important galaxies for lensing to speed up computation, as most will have negligable mass or be too far away
            \item Allows quicker lensing calculations, means we can make $R$ bigger
            \item Show `relevance` metric ($M/R^3$ but unitless)
            \item Include plot of predicted relevance vs kappa contribution??
            \end{itemize}
        \end{itemize}
    \item Lensing by Halos
        \begin{itemize}
        \item Treat each foreground halo as a separate lensing event
        \item Total lensing is the sum of each event
        \item Lookup table??
        \end{itemize}
    \item Include cartoon
    \end{itemize}
\item Checking the halo model: Use of correlation functions as summary statistics for ellipticity predictions
    \begin{itemize}
        \item Introduce the concept of correlation functions in general
            \begin{itemize}
            \item Definition
            \item Autocorrelation vs cross-correlation
            \item Use of correlation functions in cosmology
            \item Shear-shear and galaxy-galaxy correlation functions
                \begin{itemize}
                \item Reference appendix for Visualizing Correlation Functions
                \end{itemize}
            \end{itemize}
        \item Use of Jarvis`s \texttt{TreeCorr} code to calculate cf`s
        \begin{itemize}
            \item Reference appendix for Jarvis proof
        \end{itemize}
    \end{itemize}
\item Show one map with ray-traced features and lensed ellipticities vs another map with \texttt{Pangloss} features and predicted ellipticities?
\end{itemize}


Maybe do importance and/or lookup table after paper is submitted? Talk about in discussion?